\documentclass[PPFS.tex]{template/subfiles}
\begin{document}
%-----------------------------------
%	ABSTRACT
%-----------------------------------
\section*{Abstract}

Team 14 is a group of senior electrical and computer engineers at Calvin College. The group is made up of Daniel DeHoog, TJ DeVries, Paul Griffioen, and Ryan Siekman. This group was brought together to create a senior design project for Calvin College's engineering capstone course, Engineering 339 \& 340.

The Sense-Able gym is a project to give a gym the ability to turn its equipment into Internet of Things machines. The goal of this upgrade in technology is to answer a common problem for people who attend the gym. Often,  when going to the gym, a person finds his or herself  waiting in line for a machine that he or she would like to use.  With a combination of sensors and displays, along with a single
% TODO: Consisten naming for the hub
"smart hub" (referred to as the hub), the Sense-Able gym will be able to gather current use of the machines and send that to the gym's clients' standard or mobile platforms. Not only that, but clients will be able to reserve specific machines over the media of their choice. Through these two possible actions, gym clients now have the ability to lessen their wait time by checking in online or checking the busyness of the gym without waiting in line.

The Sense-Able gym will also offer administrators the ability to track usage of machines by time, by peak volume and by frequency to make executive decisions about their gym in a way that they were unable to before. A gym administrator may now see that, while they have twenty bikes, the peak usage is only eight


\end{document}
