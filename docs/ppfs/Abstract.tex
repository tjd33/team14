\documentclass[PPFS.tex]{template/subfiles}
\begin{document}
%-----------------------------------
%	ABSTRACT
%-----------------------------------
\section*{Abstract}

Team 14 is a group of senior electrical and computer engineers at Calvin College. The group is composed of Daniel DeHoog, TJ DeVries, Paul Griffioen, and Ryan Siekman. This group was brought together to create a senior design project for Calvin College's engineering capstone course, Engineering 339 \& 340.

The Sense-Able Gym is a project that provides gyms with the ability to turn their equipment into Internet of Things (IoT) machines. The goal of this upgrade in technology is to answer a common problem for people who attend the gym. Oftentimes, when going to the gym, a person finds him or herself waiting in line for a machine that he or she would like to use.  With a combination of sensors and displays (implemented using TI SensorTags), along with a single
% TODO: Consistent naming for the hub
"smart hub" (referred to as the hub and implemented using a Raspberry Pi), the Sense-Able Gym gathers current use data from machines and sends that data to the gym clients' standard or mobile platforms. In addition, clients are able to reserve specific machines over the media of their choice. Through these two possible actions, gym clients have the ability to reduce their wait time by checking in online or checking the busyness of the gym without waiting in line.

The Sense-Able Gym also offers administrators the ability to track usage of machines by time, by peak volume, and by frequency to make executive decisions about their gym in a way that they were unable to before. A gym administrator may now see that the peak usage for a particular type of machine is much less than the number of machines supplied in the gym.

\end{document}
