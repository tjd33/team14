\documentclass[PPFS.tex]{template/subfiles}
\begin{document}
%----------------------------------
%	RESEARCH
%----------------------------------
\section{Project Justification}

Research was conducted to evaluate the uniqueness of the proposed system and to understand what similar gym reservation systems exist. Throughout this research, it became clear that the majority of reservation systems that exist are systems that reserve facilities rather than systems that reserve specific pieces of gym equipment. Most of the equipment reservation systems that exist are used for reserving sports equipment, not gym machines.

However, a few systems exist that serve as reservation systems for specific machines in a gym. Two examples include the gym reservation systems for Syracuse University \cite{Syracuse} and Harvard University \cite{Harvard}. Both of these systems use machines built by Precor, a company that manufactures fitness equipment \cite{Precor}. Precor uses the fitness software provided by Preva \cite{Preva} to interface with their manufactured machines to provide a machine-specific reservation system for users.

Despite these few examples of gym reservation equipment and software, there are no known companies that supply a gym reservation system that is not already built in to each machine. The gym reservation systems provided by Precor and other companies are always built in to each workout machine. As a result, gym managers must buy this specific type of expensive equipment in order to have a reservation system for the gym. Buying this expensive equipment is often infeasible for a few reasons. If a gym manager is starting a new gym, the price of this expensive equipment is oftentimes too high. In the case of existing gyms, the benefit gained by introducing machinery with reservation capabilities is often outweighed by the expensive price of the new machinery and the fact that all the current machinery would need to be discarded.

The system that was designed in this project is unique in that it provides a gym reservation system that is general and is not built in to any specific piece of equipment. The system has the ability to be installed in any existing gym on however many machines the gym manager desires. In addition, by not being incorporated into any specific machine, this system is expected to be cheaper than the cost increase associated with upgrading from a normal machine to one with reservation system capabilities. As a result, the designed reservation system is unique and original with respect to its modularity and generality.

\end{document}