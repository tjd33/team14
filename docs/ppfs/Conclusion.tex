\documentclass[PPFS.tex]{template/subfiles}
\begin{document}
%-----------------------------------
%	CONCLUSION
%-----------------------------------
\section{Conclusion}
Due to the lack of similar products in the market and the ability to produce this product with relatively cheap hardware, this project was feasible and provides a useful and enhanced experience to gym users and administrators. Using mesh networking technology enable the networks of sensors and displays to have much greater ranges. 
Since the team ordered and received most of the hardware necessary for a basic implementation of a prototype during the first semester, the largest part of the project in the second semester was writing all the software to make all the hardware work together in harmony. Because the hub and server were accomplishing common software tasks, they were implemented without any big problems. Problems were however encountered when implementing the code for the sensors and displays, as the documentation for the SensorTags left very much to be desired. The server being the main interface with the user, it was a point where extra effort was put in to provide a great user experience and extra functionality. 

One possible future expansion on this project would be to work on the design of custom hardware for this project's purpose. Since there was no time at the end of the project once the necessary goals were accomplished, this aspect may be explored more.

There could be other ways of implementing a system that accomplishes the goals of this project, such as the initial idea of using a camera to visually analyze and determine use. In order to accomplish this project by use of visual analysis, there would need to be adequate cameras to cover the entire area of the gym, whose sizes and shapes can vary greatly. The network to connect these cameras would need to have a much higher bandwidth than what a low power network, such as the suggested implementation, would be capable of. It would not be feasible to run the cameras on batteries, so additional wired installation would be necessary. The amount of computational power needed to do the analysis on visual data would be much greater than what sensors require. There could be ways to also accomplish the goals by using the gym user's smart phone. In order to implement this system, it would first be necessary for each user to have a smart phone. Each user would then also have to download the application, enter information, and opt into the reservation and sensing system. This would make the system very alienating to those who are only visiting the gym or are not comfortable with smart phone technology. The reliability of such a system would also be very inconsistent unless the user is asked what specific machine they are using, which would be bothersome, as it would have difficulty in determining the exact machine in use. Sensors and the IoT is a rapidly growing industry, and the method of implementation chosen takes advantage of many technological devices and advances currently available in the market.

\end{document}
