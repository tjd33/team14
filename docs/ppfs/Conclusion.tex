\documentclass[PPFS.tex]{template/subfiles}
\begin{document}
%-----------------------------------
%	CONCLUSION
%-----------------------------------
\section{Conclusion}
Due to the lack of a similar products in the market and the ability to produce this product with relatively cheap hardware this project is feasible and will provide a useful and enhanced experience to gym users and administrators. It will be necessary for the network technology used by the sensors and displays to be a mesh network in order to have the necessary range. The server and hub don't have any immediate problems as they will be implemented largely though software. This project contains a large amounts of software and that will likely be a point which has the potential to provide a great user experience and added functionality beyond the basic goals. Since the team has already ordered and received most of the hardware necessary for a basic implementation of a prototype, the largest part of the project ahead is writing all the software to make all the hardware work together in harmony. 

One possible future expansion on this project would be to work on the design of custom hardware for this projects purpose. This aspect may be explored more if there is adequate time towards the end of the project, once the necessary goals have been accomplished. 

There could be other ways of implementing a system that accomplishes the goals of this project, such as the initial idea of using a camera to visually analyze and determine use. In order to accomplish this project by use of visual analysis, there would need to be adequate cameras to cover the entire area of the gym, whose sizes and shapes can vary greatly. The network to connect these cameras would need to be much higher bandwidth than what a low power network, such as the suggested implementation, would be capable of. It would not be feasible to run the cameras on batteries, so additional wired installation would be necessary. The amount of computational power needed to do the analysis on visual data would be much greater than what sensors require. There could be ways to also accomplish the goals by using the gym user's smart phone. In order to implement this system it would first be necessary for each user to have a smart phone. Each user would then also have to download the application, enter information, and opt into the reservation and sensing system. This would make the system very alienating to those who are only visiting the gym, or are not comfortable with smart phone technology. The reliability of such a system would also be very inconsistent, unless the user is asked what specific machine they are using, which would be bothersome, as it would have difficulty in determining the exact machine in use. Sensors and the IoT is a rapidly growing industry and the method of implementation chosen takes advantage of many technological devices and advances currently available in the market. 

\end{document}
