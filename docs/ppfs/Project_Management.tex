\documentclass[ppfs.tex]{template/subfiles}
\begin{document}
%------------------------------------------
%	PROJECT MANAGEMENT
%------------------------------------------
\section{Project Management}
    \subsection{Team Organization}
	Team 14 consists of Daniel DeHoog, Paul Griffioen, TJ DeVries, and Ryan Siekman. The work on the project thus far has been largely research and in that aspect the work has been split up primarily by the different components of the project. Daniel has done the most research and lead the decisions concerning the displays. Paul has been in charge of the sensor research. TJ has already started on the server and has made the overall system diagrams for how all of these components will communicate and work together. Ryan has done the research concerning the hub component. Beyond this split of work, TJ has been in charge of our schedule worksheet, while Paul wrote up the submission for the Eric DeGroot fund proposal. Ryan has maintained meeting minutes and action items. All of the members of the team have been in communication with various people outside our direct team, such as emails to our professors, or other's we work with. These tasks have been assigned more on a basis of who it is relevent to, rather than having a single person do all of the email communication. 
	Team meetings occur every Monday evening from 3:30pm to 4:30pm after senior design. They run by first covering what the previous action items where and checking if they were completed or if they will continue to be an action item for the following week. Then any items that are due soon are addressed, and last all the future action items are assigned. This last action is usually done while discussing what the next steps of the project need to be and the correct order and importance of them, while refering to the schedule as well.
	Documents related to our senior design project are all kept in either Github or Google Drive. Any documents or work internal to the team is on Google Drive, while all of the documents and reports that will be turned in and all of the code written thus far for the project are kept in GitHub. The reason for keeping all of the reports and documents that will be distributed outside our team on GitHub is because these are all being written in LaTex. This provides much more flexibility in the formatting and combining of documents written by the individuals on the team, along with version control to see what has been changed and when. The shared Google Drive folder contains the teams meeting minutes, schedule, research notes, presentation, and budget.
	
    \subsection{Schedule}
	Initially in trying to find a scheduling system that would work well, various options were tested, including simple calendar integration with slack, trello, asana, and Microsoft Project. The last one, Project, is what the initial schedule was created with. However due to its lack of online accessibility, it was exported and saved in our Google Drive folder as a spreadsheet. From there TJ has continued to add functionality to it in order for us to be able to track our hours and enable better visualization of tasks and their due dates. The schedule gets updated with events during the team meetings or as they are assigned. Each team member edits the schedule as they spend more hours on tasks, or complete them. The schedule determines what action items are assigned and what parts of the project are most critical to work on at any given time. When schedule issues arrise, it is usually a matter of spending more time on tasks, delaying a due date, or deciding that one item must precede another. These issues are dealt with as they arrise and there have yet to be any large issues with scheduling. However the largest issue has been simply ensuring the schedule works, the functions within the spreadsheet that allow the hour tracking and views are difficult to maintain when items are added, moved around, or viewed on other sheets. 
	
    \subsection{Budget}
    The budget is maintained through a spreadsheet in the shared Google Drive folder the team uses. It is maintained by Ryan and thus far has only been needed to be updated once, as the team has only made one order. It is necessary to update the budget spreadsheet as orders are made, or information on different pricing is received. The budget hasn't been largelly used as a management tool as it doesn't largely contrict the ability to make a prototype of the system. It only limits how large of a prototype that could be implemented. As the hope is to test the system at Calvin's gym, the budget is large enough to monitor at least two pieces of gym equipment. Any increase in the budget would allow the team to increase this testing or to test with different components than initially selected. This far in the project no budget issues have arrisen, however if they do then decisions will be made concerning what purchases will be more crititcal to our prototype than others. These decisions can be made by determining what the impact of one decision will be versus another, the primary problem which may arise being between expanding the prototype system or testing a larger variety of hardware for the system. 
	
	\subsection{Method of Approach}
	The path that led to this project started with trying to determine a useful project that all members of the team would be interested in. This led to several options, most related to the internet of things (IOT). The problem at this point became finding a project or potential product idea that wasn't currently on the market, would be feasible for the scope of senior design, and something that all members of the team would be interested in. Paul initially thought of an app to show users what gym equipment is in use, as gym users can get frustrated when a machine they would like to use is constantly in use. The initial idea was to accomplish this with visual analisys of the gym through a camera. Using imaging technology to determine where people were located and what machines they were using. Due to the large variation in gym sizes and set ups, it was determined that this would be difficult to accomplish generically for various gyms. With the large increase in IOT technology and the increasing number of sensors being used for various tasks, using sensors to determine the equipment use seemed like a more feasible idea. The next stage was what would make this more useful to the user. Being able to not just sense whether the equipment is occupied or not doesn't completely solve the issue of never having open equipment, as it would then just let the user know it's always occupied. Adding a reservation aspect to the system allows the user to ensure that they will have time on the desired equipment, and let other people know when it is reserved. In order to show in the gym whether equipment is reserved or not it now became necessary to add some type of display or notification ability to the equipment. The use of a hub and server developed as the backend necessary to support the usability of this system then became necessary. Although it could be feasible to have the hub located at the gym be more powerful and store the analytical data necessary, this would make the tech support of the whole system much more difficult and the current course of technology is to host in the cloud and provide access to systems remotely, which would be much more difficult if the hub and server were combined in a single piece of equipment and located only at the gym. 
	
\end{document}