\documentclass[PPFS.tex]{template/subfiles}

\begin{document}
%--------------------------------------------------------
%       TASK SPECIFICATIONS AND SCHEDULE
%--------------------------------------------------------
\begin{comment}
These are the requirements as stated in the template given by Profess M.

    Detailed task specifications are a key part of your design.  Tasks that take more than a day or two should be broken down into sub-tasks whenever possible.  Typically tasks specs and schedule are combined since they are intimately related.  The percentage complete on each task should be included in the PPFS.  Address any issues related to tasks that are behind schedule in the project management section.  Organize tasks hierarchically, such as system architecture, mechanical, electrical (hardware, software), etc.  Provide block diagrams to help explain how you divided the project into various phases/parts/categories.  
    Many teams make the mistake of thinking the prototype is the design.  This is rarely the case.  Only a few teams produce a single device.  Most are designing a product that can be mass produced.  Thus the prototype is your method of validating your production design.   It is a major test or validation activity on the way to completing the design.  Include prototyping activities as tasks that help you finish the production design.  
    Summarize your schedule by reporting the total expected person-hours needed to complete the design.  The best reports track the schedule progress by reporting the month by month estimate of total person-hours and graphing how it changed as you gained experience.
\end{comment}

\section{Task Specifications and Schedule}
The team met at the beginning of the semester to brainstorm what the project should look like, and then the logical steps no how to accomplish these goals. After creating a list of requirements, considering both goals of the senior design project and deliverables for the senior design class, tasks were created and assigned. These tasks were scheduled throughout the semester so as to make sure there would be ample time for revision, handling any problems that arose and to ensure quality work on all aspects of the project.

\subsection{Organization of Tasks}
Tasks were organized in a functional manner. As mentioned above, the group considered what the requirements of the project would be, and using those requirements continued to break down the project tasks into smaller and smaller components, until they were easily described and able to be tested if they were done or not. This allowed the team to have a clear outlook on what tasks needed to be done each week, and each team member was able to work effectively because of it. 

% Example breakdown
\subsubsection{PPFS}
The PPFS was broken down by section and each section was assigned a due date for a rough draft to be completed. Once all of the sections had been completed, the group met to work through any questions any members had on individual sections. Following this, the team had several revision meetings.

\subsubsection{Research}
The research was split into four different sections, one for each piece of hardware and the respective software for each section.

\subsubsection{Hardware}
Paul researched the sensor and mesh network aspects of the project. Dan researched the display aspect of the project. Ryan researched the hub aspect of the project. TJ researched the server aspect of the project, along with contributing to the display and sensor research. After Paul and Dan completed their research, they were tasked with creating a Bill of Materials for what parts should be ordered for the project. Once the research was completed, the group had a meeting to order parts.

\subsubsection{Software}
-- This has not been researched yet because we are waiting to find out what is happening with the devpack for the TI sensortags --

\subsubsection{Oral Presentations}
Ryan and TJ were tasked with doing the Oral Presentations. Initially they spent some time brainstorming the main goals of the project. Then they created the presentation for Oral Presentation I. They were also tasked with the second Oral Presentation. --- Guidelines have not yet been released for what is required for this ---

\subsubsection{Website}
TJ was assigned webmaster by the group. He then was tasked with finding a good template to use. Following that task, he edited the template to meet the needs of the Senior Design team. Then he was tasked with launching the site.

\subsubsection{Management}
Each week, the entire team was tasked with a meeting on Monday afternoons. Otherwise, TJ was tasked with maintaining the schedule. Ryan was tasked with handling any paperwork or billing for project materials.

% Show or summarize the heirarchy?

\subsection{Summary of Tasks}
After creating functional tasks based on the requirements of the project, the team was able to do analysis on what would be expected throughout the Fall Semester. A Gantt chart was created to visualize the progress of the project and to view critically linked tasks.
% TODO: Don't forget this
This gantt chart can be found -- Insert link to gantt chart once complete --.

\subsubsection{Expected Hours By Month}
However, the group was able to do more than just view progress, but because of the task breakdown, each task was able to have estimated hours attached to it. When these hours were considered as a sum, it was possible to see what the expected hours per month would be.

% TODO: Don't forget this
-- Insert final expected hours per month table, not complete because waiting on devpack problems  --
% Here is where we insert a summary of the hours 

\subsubsection{Actual Progress By Month}

% Show some of our reports by month, detailing both objectives and what each person accomplished.
% TODO: Don't forget this.
--- Analysis of the final graph when it is finished will be shown below ---

\begin{figure}[H]
    \centering
    \includegraphics[width=\textwidth]{work_by_day.png}
    \caption{Work Done by Day}
\end{figure}

\end{document}
