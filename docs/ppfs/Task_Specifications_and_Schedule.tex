\documentclass[ppfs.tex]{template/subfiles}

\begin{document}
%--------------------------------------------------------
%       TASK SPECIFICATIONS AND SCHEDULE
%--------------------------------------------------------
\begin{comment}
These are the requirements as stated in the template given by Profess M.

Detailed task specifications are a key part of your design.  Tasks that take more than a day or two should be broken down into sub-tasks whenever possible.  Typically tasks specs and schedule are combined since they are intimately related.  The percentage complete on each task should be included in the PPFS.  Address any issues related to tasks that are behind schedule in the project management section.  Organize tasks hierarchically, such as system architecture, mechanical, electrical (hardware, software), etc.  Provide block diagrams to help explain how you divided the project into various phases/parts/categories.  
Many teams make the mistake of thinking the prototype is the design.  This is rarely the case.  Only a few teams produce a single device.  Most are designing a product that can be mass produced.  Thus the prototype is your method of validating your production design.   It is a major test or validation activity on the way to completing the design.  Include prototyping activities as tasks that help you finish the production design.  
Summarize your schedule by reporting the total expected person-hours needed to complete the design.  The best reports track the schedule progress by reporting the month by month estimate of total person-hours and graphing how it changed as you gained experience.
\end{comment}

\section{Task Specifications and Schedule}

\subsection{Organization of Tasks}

% Example breakdown
\subsubsection{PPFS}
\subsubsection{Research}
\subsubsection{Hardware}
\subsubsection{Software}
\subsubsection{Oral Presentations}
\subsubsection{Website}
\subsubsection{Management}
\subsubsection{Various Side Projects}

% Show or summarize the heirarchy?

\subsection{Summary of Tasks}

% Provide references to the gantt chart?

\subsubsection{Expected Hours By Month}

% Here is where we insert a summary of the hours 

\subsubsection{Actual Progess By Month}

% Show some of our reports by month, detailing both objectives and what each person accomplished.

\end{document}