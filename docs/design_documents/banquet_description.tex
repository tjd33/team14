\documentclass[letterpaper,11pt]{./templates/texMemo} % Set the paper size (letterpaper, a4paper, etc) and font size (10pt, 11pt or 12pt)

\usepackage{graphicx}
\graphicspath{ {../img/} }
\usepackage[parfill]{parskip} % Adds spacing between paragraphs
\setlength{\parindent}{0pt} % Indent paragraphs
\pagenumbering{gobble} % Eat the page numbers

% Begin writing the document
\begin{document}

\centerline{\Huge{\textbf{Team 14: Sense-Able Gym}}}
\centerline{\LARGE{Daniel DeHoog, TJ DeVries, Paul Griffioen, Ryan Siekman}}

% Team Member Photo
\centerline{\includegraphics[width=\textwidth]{Team.jpg}}

Team 14, the Sense-Able Gym, is composed of four electrical/computer engineering students with interests in computer science. The team set out to create a system that allows people, both clients and administrators, to interact with gyms in a modern and smart way. This system addresses the problem of overcrowded gyms and allows clients and administrators to make more efficient use of gyms.

The system allows users to view what machines are currently being used through a web interface, and it allows users to reserve machines for personal use, eliminating machine wait times in busy gyms. The system also provides gym administrators with the ability to understand more clearly what machines get used along with how often and when they are used. This data allows the gym manager to make more informed decisions when buying and placing gym equipment.

The system has a low cost and is reliable, having been tested thoroughly. It consists of small sensors and displays (for showing reservations on specific machines) that can be placed on any machine in any gym. As a result, the system is unique in its generality and modularity, enabling it to be installed in any gym and allowing the gym manager to choose which machines should be part of the system.

\end{document}
